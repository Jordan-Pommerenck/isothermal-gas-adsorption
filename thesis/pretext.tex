%\thispagestyle{plain}
\thispagestyle{empty}
\begin{center}
	\large
	AN ABSTRACT OF THE THESIS OF
\end{center}


\justify{}
\underline{Jordan K. Pommerenck} for the degree of \underline{Doctor of Philosophy} in \underline{Physics}
presented on April 30, 2020
\vspace{1.0cm}

\justify{}
Title: \underline{Advancing renewable gas storage using flat-histogram methods}.
\vspace{2.0cm}

\justify{}
Abstract approved:
\underline{\hspace{11cm}}
\vspace{0.5cm}

%\centering{}
\hspace{7.0cm} David J. Roundy
\vspace{1.0cm}

\justify{}
\doublespacing
This work introduces the development of the novel flat-histogram Monte
Carlo (MC) method SAD and explores the convergence properties of a
variety of related weight-based MC methods. The new method is applied
to a variety of physical ‘test’ systems including the 2D Ising model,
square-well fluid, and Lennard-Jones clusters. A driving motivation for
developing novel Monte Carlo methods that have physically based tunable
parameters is to provide simulation methods for exploring metal-organic
frameworks (MOFs). A theoretical framework for gas adsorption and
delivery in metal-organic frameworks is developed and compared with
experimental data. SAD is introduced as a powerful method for examining
thermodynamic properties for MOFs. A preliminary groundwork is laid for
the future development of a multi-dimensional SAD which would be able
to compute all isotherms for any given MOF thereby providing a powerful
way for researchers to determine a MOF’s suitability for gas storage
applications.


\newpage{}
\thispagestyle{empty}
\singlespacing

\vspace*{4.0cm}
\begin{center}
$\copyright$ Copyright by Jordan K. Pommerenck \\
April 30, 2020 \\
(All Rights Reserved)
\end{center}

\newpage{}
\thispagestyle{empty}
\singlespacing

\begin{center}
Advancing renewable gas storage using flat-histogram methods

\vspace{1.0cm}
by \\
Jordan K. Pommerenck \\
\vspace{3.0cm}
A DISSERTATION \\
\vspace{0.5cm}
submitted to \\
\vspace{0.5cm}
Oregon State University \\
\vspace{3.0cm}
in partial fulfillment of \\
the requirements for the \\
degree of \\
\vspace{1.0cm}
Doctor of Philosophy \\
\vspace{3.0cm}
Presented April 30, 2020  \\
Commencement June 2020
\end{center}

\newpage{}
\thispagestyle{empty}
\singlespacing
\justify{}
\underline{Doctor of Philosophy} dissertation of \underline{Jordan K. Pommerenck} presented on
April 30, 2020. \\

\justify{}
\vspace{0.5cm}
APPROVED: \\

\justify{}
\underline{\hspace{15cm}}
\begin{flushleft}
Major Professor, representing Physics
\vspace{1.0cm}
\end{flushleft}

\justify{}
\underline{\hspace{15cm}}
\begin{flushleft}
Chair of the Department of Physics
\vspace{1.0cm}
\end{flushleft}

\justify{}
\underline{\hspace{15cm}}
\begin{flushleft}
Dean of the Graduate School
\vspace{1.0cm}
\end{flushleft}

\vspace{3.0cm}
\justify{}
I understand that my dissertation will become part of the permanent collection
of Oregon State University libraries. My signature below authorizes release of my dissertation to any reader upon request. \\
\vspace{0.5cm}
\begin{flushleft}
\underline{\hspace{15cm}}
\end{flushleft}
\centering{}
Jordan K. Pommerenck, Author

\newpage{}
\thispagestyle{empty}
\begin{center}
	\large
	ACKNOWLEDGEMENTS
\end{center}
\justify{}
\doublespacing
I gratefully acknowledge my graduate advisor David Roundy for his
constant support and guidance throughout my research. His breadth of knowledge
and hands-off approach have tremendously helped me to develop as a scientific
researcher.
I express thanks to my graduate committee members: David McIntyre, Davide Lazzati, Yun-Shik Lee, and GCR Chih-hung (Alex) Chang.
I would like to acknowledge the many outstanding graduate students that have positively impacted me (too many to name here but you know who you are).
I would like to acknowledge some of the outstanding undergraduate students
from our research lab that have worked on related research:
Tanner Simpson, Cade Trotter, and Christopher May.
Thank you all for your helpful discussions related to research and life in
general.
Finally, I would like to thank my parents for their unconditional support
throughout my PhD research.
