\chapter{General Conclusion}

In the first manuscript \emph{Stochastic approximation Monte Carlo with a dynamic update factor}, we developed and refined the novel Monte Carlo method SAD. We compared the convergence properties of SAD to a variety of other weight-based flat-histogram methods. We also examined the convergence behavior applied to the square-well fluid and a Lennard-Jones cluster.

In the second manuscript \emph{Flat-histogram method comparison on 2D Ising model}, we tested SAD on the 2D Ising model against a number of weight-based methods including a ``production'' run WL. We found that even for systems where it is convenient to choose an energy range, SAD performs incredibly well.

In the third manuscript \emph{An upper bound to gas delivery via pressure-swing adsorption in nanoporous materials}, we develop an upper bound on gas delivery and compare the theory with experimental data. We found that while hydrogen may not be viable at room temperature, methane storage can potentially meet the Department of Energy (DOE) targets. The theory also suggests an ideal energy of attraction which will help when determining what MOFs to simulate.

Each of these manuscripts laid the groundwork for the development of 2D SAD. SAD has been tested on a variety of physical systems and shows tremendous promise for simulating isotherms. In addition, an upper bound was developed that will aid in simulating MOFs using 2D Sad.

There are two important steps that remain to complete the extension.  The first is the development of SAD with a minimum particle number $N_{\min}$ instead of a $T_{\min}$. By varying the number of particles, the simulation can yield thermodynamic insights for all densities. The second step that remains is the merger of each SAD method such that a multidimensional 2D SAD is formed. Such a Monte Carlo method would be an invaluable tool for driving research in the gas storage field.