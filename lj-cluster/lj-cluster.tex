\documentclass[letterpaper,twocolumn,amsmath,amssymb,pre,aps,10pt]{revtex4-1}
\usepackage{graphicx} % Include figure files
\usepackage{color}
\usepackage{nicefrac} % Include for inline fractions

\usepackage{xargs}                      % Use more than one optional parameter in a new commands
\usepackage[pdftex,dvipsnames]{xcolor}  % Coloured text etc.
\usepackage[colorinlistoftodos,prependcaption,textsize=normalsize]{todonotes}
\usepackage{mdframed}

% define colors for comments
\definecolor{dark-gray}{gray}{0.10}
\definecolor{light-gray}{gray}{0.70}

\newcommand{\red}[1]{{\bf \color{red} #1}}
\newcommand{\green}[1]{{\bf \color{green} #1}}
\newcommand{\blue}[1]{{\bf \color{blue} #1}}
\newcommand{\cyan}[1]{{\bf \color{cyan} #1}}

\newcommand{\davidsays}[1]{{\color{blue} [\green{David:} \emph{#1}]}}
\newcommand{\jpsays}[1]{{\color{red} [\blue{Jordan:} \emph{#1}]}}
\newcommandx{\jpcom}[2][1=inline]{\todo[linecolor=gray,backgroundcolor=light-gray,bordercolor=dark-gray,#1]{\textbf{Jordan says:} #2} }
\begin{document}

\title{SAD LJ-Cluster paper
}

\author{Jordan K. Pommerenck} \author{David Roundy}
\affiliation{Department of Physics, Oregon State University,
  Corvallis, OR 97331}

\begin{abstract}
  We study the impact of the Lennard-Jones cluster `constraining' radius on 
  the thermodynamic properties such as heat capacity. We use SAD.
  % We present a new Monte Carlo algorithm based on the Stochastic
  % Approximation Monte Carlo (SAMC) algorithm for directly calculating
  % the density of states. The proposed method is Stochastic
  % Approximation with a Dynamic update factor (SAD)
  % which dynamically adjusts the update factor $\gamma_t$ during the course of
  % the simulation. We test this method on the 31-atom Lennard-Jones cluster and
  % compare the convergence behavior of several
  % related
  % Monte Carlo methods. We find that both the SAD and $1/t$-Wang-Landau ($1/t$-WL)
  % methods rapidly converge to the
  % correct density of states without the need for the user to specify an
  % arbitrary tunable parameter $t_0$ as in the case of SAMC.  SAD requires
  % as input the temperature range of interest, in contrast to
  % $1/t$-WL, which requires that the user identify the interesting range
  % of energies.
  % %
  % The convergence of the $1/t$-WL method is very sensitive to the energy
  % range chosen for the low-temperature heat capacity of the
  % Lennard-Jones cluster.
  % %
  % Thus, SAD is more powerful in the common case in which the range
  % of energies is not known in advance.
\end{abstract}

\maketitle

\section{Introduction}
% (1) After examination, a major triumph would be to determine $R_c$ theoretically particularly
% for large constraining radii i.e. extract volume independent heat capacity.

% (3) 2D histogram could give information out of all clusters for a single
% simulation (will it converge?).

Lennard-Jones (LJ) clusters have been extensively studied using both Monte Carlo methods~\cite{frantsuzov2005size, mandelshtam2006structural,
mandelshtam2006multiple} and molecular dynamics~\cite{honeycutt1987molecular, calvo1995configurational, calvo2000phase} as global optimization
problems~\cite{wales1997global, wales1998global, wales1999global, doye1999double} and systems with finite size phase
transitions~\cite{neirotti2000phase, sabo2005pressure, sehgal2014phase}.

Over the last several decades, the global minima of various clusters with a number of atoms less than 1000 have been
found using simulation techniques such as \jpsays{cite these two!  Xueguang Shao and DJ WALES} basin-hopping and genetic algorithms.
The heat capacity curves for various clusters are used to predict solid-solid and liquid-solid transitions

% The latent heat of these transitions was computed as the area under each peak in the heat capacity curve. It was found that the
% latent heat associated with the first peak is approximately 0.67lm, where lm = 0.29/Nǫ is the latent heat of the second peak,
% which, as we will show below, signals the melting of the cluster. (N = 309  https://arxiv.org/pdf/cond-mat/0512147.pdf)


talk about histogram methods here and maybe parallel tempering to lead up to rc for simulation!

A common practice to prevent cluster dissociation via boil-off is to place the atoms in a `constraining'
sphere of a given radius. This presents critical problems in terms of accurately simulating thermodynamic
properties. Ideally, the user would like the radius to be as small as possible to decrease simulation time.
Unfortunately, too small of a radius for a given number N atoms results in incorrect simulation properties
such as shifted heat capacity peaks. Likewise, too large a radius results in longer simulation times (IS THIS TRUE?)
due to the time necessary to discover all energy states due to the atoms to equilibrating.

Early research on LJ clusters typically chose a small constraining radius to achieve convergence in
a reasonable simulation time. Unfortunately, later research confirmed that these radii were too small
to ensure accurate results but noted that they chose their radii empirically. The reason for this is that
it is hard to choose a constraining sphere for any given LJ simulation that achieves ergodicity while not
introducing fluctuations in the desired thermodynamic properties.

In this work, we explore the impact of the `constraining' sphere on heat capacity calculations using flat-histogram
methods.

(Need to fix lj-comparison.py to plot data probably manually is best by command line with argparse)
N = 38 for the radius comparison.

% \section{Literature Review}

% \jpsays{https://aip.scitation.org/doi/pdf/10.1063/1.481671}
%
% Forty distinct temperatures have been used in the parallel tempering simulations
% of LJ38 ranging from $T = 0.0143 \epsilon/k_B$ to $T = 0.337 \epsilon/k_B$ . The
% simulations have been initiated from random configurations of the 38 atoms
% within a constraining sphere of radius 2.25$\sigma$. We have chosen $R_c =
% 2.25\sigma$, because we have had difficulties attaining ergodicity with larger
% constraining radii. With large constraining radii, the system has a significant
% boiling region at temperatures not far from the melting region, and it is
% difficult to execute an ergodic walk with any method when there is coexistence
% between liquidlike and vapor regions. Constraining radii smaller than
% $2.25\sigma$ can induce significant changes in thermodynamic properties below
% the temperature of the melting peak. Using the randomly initialized
% configurations the initialization time to reach the asymptotic region in the
% Monte Carlo walk has been found to be long with about 95 million Metropolis
% Monte Carlo points followed by 190 million parallel tempering Monte Carlo points
% included in the walk prior to data accumulation. This long initiation period can
% be made significantly shorter by initializing each temperature with the
% structure of the global minimum.
%
% \jpsays{https://aip.scitation.org/doi/pdf/10.1063/1.1397329}
%
% This paper acknowledges that previous paper (mentioned above) is too small for large
% cluster radius but doesn't give a good way to choose Rc
%
% \jpsays{https://www.sciencedirect.com/science/article/pii/S0010465512000859}
%
% For water clusters a hard wall is used to prevent evaporation.  Landau notes that for
% N > 10 it is hard to choose Rc.  He uses `pressure' NPT ensemble:
% $\sigma^{*}=3.154$ Angstroms for TIP4 and they use 0.1 ${\sigma^{*}}^{3}$ as a metric to
% prevent evaporation (they got it empirically)

% \section{Conclusions}
% We have introduced a new algorithm, a variant of the SAMC method, which

\bibliography{lj-cluster}% Produces the bibliography via BibTeX.

\end{document}
