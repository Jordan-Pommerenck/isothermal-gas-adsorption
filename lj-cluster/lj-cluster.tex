\documentclass[letterpaper,twocolumn,amsmath,amssymb,pre,aps,10pt]{revtex4-1}
\usepackage{graphicx} % Include figure files
\usepackage{color}
\usepackage{nicefrac} % Include for inline fractions

\usepackage{xargs}                      % Use more than one optional parameter in a new commands
\usepackage[pdftex,dvipsnames]{xcolor}  % Coloured text etc.
\usepackage[colorinlistoftodos,prependcaption,textsize=normalsize]{todonotes}
\usepackage{mdframed}

% define colors for comments
\definecolor{dark-gray}{gray}{0.10}
\definecolor{light-gray}{gray}{0.70}

\newcommand{\red}[1]{{\bf \color{red} #1}}
\newcommand{\green}[1]{{\bf \color{green} #1}}
\newcommand{\blue}[1]{{\bf \color{blue} #1}}
\newcommand{\cyan}[1]{{\bf \color{cyan} #1}}

\newcommand{\davidsays}[1]{{\color{blue} [\green{David:} \emph{#1}]}}
\newcommand{\jpsays}[1]{{\color{red} [\blue{Jordan:} \emph{#1}]}}
\newcommandx{\jpcom}[2][1=inline]{\todo[linecolor=gray,backgroundcolor=light-gray,bordercolor=dark-gray,#1]{\textbf{Jordan says:} #2} }
\begin{document}

\title{Ideal choice of a constraining sphere in Lennard-Jones cluster
simulations using flat-histogram methods
}

\author{Jordan K. Pommerenck} \author{David Roundy}
\affiliation{Department of Physics, Oregon State University,
  Corvallis, OR 97331}

\begin{abstract}
  We study the impact of the Lennard-Jones cluster `constraining' radius on 
  the thermodynamic properties such as heat capacity. We use SAD. We examine
  the 38 atom Lennard-Jones cluster.
\end{abstract}

\maketitle

\section{Introduction}
% (1) After examination, a major triumph would be to determine $R_c$ theoretically particularly
% for large constraining radii i.e. extract volume independent heat capacity.

% (3) 2D histogram could give information out of all clusters for a single
% simulation (will it converge?).

Lennard-Jones (LJ) clusters have been extensively studied using both Monte Carlo methods~\cite{frantsuzov2005size, mandelshtam2006structural,
mandelshtam2006multiple} and molecular dynamics~\cite{honeycutt1987molecular, calvo1995configurational, calvo2000phase} as global optimization
problems~\cite{wales1997global, wales1998global, wales1999global, doye1999double} and systems with finite size phase
transitions~\cite{neirotti2000phase, sabo2005pressure, sehgal2014phase}.

Over the last several decades, the global minima of various clusters with a number of atoms less than 1000 have been
found using simulation techniques such as \jpsays{cite these two!  Xueguang Shao and DJ WALES} basin-hopping and genetic algorithms.
Other Monte Carlo methods such as flat-histogram algorithms~\cite{?} and parallel tempering~\cite{?} have been used to provide insight to 
the thermodynamic properties of small clusters such as heat capacity and latent heat. The heat capacity curves for various clusters can
then be used to predict solid-solid and liquid-solid transitions in the cluster. 

% The latent heat of these transitions was computed as the area under each peak in the heat capacity curve. It was found that the
% latent heat associated with the first peak is approximately 0.67lm, where lm = 0.29/Nǫ is the latent heat of the second peak,
% which, as we will show below, signals the melting of the cluster. (N = 309  https://arxiv.org/pdf/cond-mat/0512147.pdf)

% Basin hopping apparently uses a MC simulation with an rc radius but only used 
A difficulty in Monte Carlo simulation of LJ clusters is the choice of a `constraining' radius $R_c$ necessary to prevent cluster dissociation.
The choice of an ideal $R_c$ presents critical problems in terms of accurately simulating thermodynamic
properties. Ideally, the user would like the radius to be as small as possible to decrease simulation time.
Unfortunately, too small of a radius for a given number N atoms results in incorrect simulation properties
such as shifted heat capacity peaks. Likewise, too large a radius results in longer simulation times
due to an increase in the time necessary to discover all energy states due to atom equilibration. The random
sampling of the energies can become increasingly difficult with the addition of the liquid-like and vapor regions~\cite{neirotti2000phase}.

Some research on LJ clusters choose a small constraining radius to achieve convergence in
a reasonable simulation time~\cite{neirotti2000phase}. Unfortunately, later research~\cite{frantz2001magic} confirmed that these radii can be to small
to ensure accurate results for arbitrary cluster sizes but noted that they chose their radii empirically. Additional work has been done to determine
an ideal $R_c$ for atomic clusters~\cite{yin2012massively} but for particle numbers much larger than $N=10$, the task of choosing an appropriate radius is complicated and empirical
in nature. 

The reason for this is that
it is hard to choose a constraining sphere for any given LJ simulation that achieves ergodicity while not
introducing fluctuations in the desired thermodynamic properties.

In this work, we explore the impact of the `constraining' sphere on heat capacity calculations using flat-histogram
methods.

\jpsays{(Need to fix lj-comparison.py to plot data probably manually is best by command line with argparse)
N = 38 for the radius comparison.}

% \section{Methods}

% \section{Results}

% \section{Conclusions}


\bibliography{lj-cluster}% Produces the bibliography via BibTeX.

\end{document}
